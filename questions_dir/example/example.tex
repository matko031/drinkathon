\documentclass[10pt,a4paper]{article}
\usepackage[utf8]{inputenc}
\usepackage{amsmath}
\usepackage{amsfonts}
\usepackage{amssymb}
\usepackage{listings}
\begin{document}

\title{Example: select letters}
\maketitle



\paragraph{Task\\ \\}

Write a programme that will print a certain letter from each word in a series of words.

\paragraph{Input\\ \\}

Input consists of a number of words together with a certain number for each words that specifies which letter from the word needs to be printed. In the first row, there is one number n specifying the total number of words. After that come n rows with one number and one word per row, separated by a space. The number specifies which letter from the word needs to be printed. You can assume that the number will never be bigger than the number of letters in each word.


\begin{center}
Example input
\hrule
\end{center}
\mbox{}\\
4\\
4 programming\\
13 implementation\\
3 winning\\
1 team\\
\hrule

	
\paragraph{Output\\ \\}

The output consists of n rows where in each row the selected letter needs to be printed.

\begin{center}
Example output
\hrule
\end{center}
\mbox{} \\
g\\
o\\
n\\
t\\
\hrule


\newpage

\begin{lstlisting}[language=Python]
# input() reads one input line as a string
# int(x) will cast x to an int
letters = int(input())

# for i in range(N) is a for loop where goes from 0 to (and including) N-1
for i in range(letters):
	# everything with this level of indentation belongs to the for loop 
	# 	(there are no {} in python)

	# input() will read one lien from the standard input
	# .rstrip() will strip the extra spaces from the end of a string
	# .split() will split the string into a list with space 
	# 	as default separators
	row = input().rstrip().split()
	
	# save 1st element (index 0) of the list row to variable position
	position = row[0]
	# save 2nd element (index 1) of the list row to variable position
	word = row[1]
    
    # print(x) will print whatever x is
    print(word[int(position) - 1])
\end{lstlisting}




\end{document}